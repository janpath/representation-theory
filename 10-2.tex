% Copyright (c) Lars Niedorf, Jan Path 2018
%
% This work is licensed under the Creative Commons Attribution-ShareAlike 4.0
% International License. To view a copy of this license, visit
% http://creativecommons.org/licenses/by-sa/4.0/ or send a letter to Creative
% Commons, PO Box 1866, Mountain View, CA 94042, USA.

% !TEX root = main.tex

% 11-01-2018

%--------------------------------------------------------------------------------------------------------------------

Let $\Lambda$ be a finite-dimensional algebra over an algebraically closed field $k$.
We denote $Q_\Lambda$ the Gabriel quiver of $\Lambda$. Let $k[Q_\Lambda]$ be the path-algebra of
$Q_\Lambda$. For $n \subseteq \mathbb{N}_0$ let $k[Q_\Lambda]_{\geq n}$ be the subspace of $k[Q_\Lambda]$ generated by
all paths of length $\geq n$.

%--------------------------------------------------------------------------------------------------------------------

\begin{theorem}[Gabriel]\label{3.2.7}
  Let $\Lambda$ be a finite-dimensional basic algebra over $k$. Then there is an
  isomorphism
    \[ \Lambda \isomorphic \factor{k[Q_\Lambda]}{I} \]
  where $I \trianglelefteq k[Q_\Lambda]$ is an ideal such that $k[Q_\Lambda]_{\geq n} \subseteq I \subseteq
  k[Q_\Lambda]_{\geq 2}$ for some $n \geq 2$.
\end{theorem}

%--------------------------------------------------------------------------------------------------------------------

\begin{remark}
In the proof we will make use of the following version of Proposition~I.\ref{1.2.8}: \textit{If $\Lambda$ is a finite-dimensional $k$-algebra and $\Lambda' \subseteq \Lambda$ a subalgebra such that $\Lambda = \Lambda' + \Rad^2(\Lambda)$, then $\Lambda = \Lambda'$.}
\end{remark}

%--------------------------------------------------------------------------------------------------------------------

\begin{proof}
  Let $\{ \varepsilon_1, \ldots, \varepsilon_r \}$ be the set of paths of length zero. By assumption,
  there exists orthogonal primitive idempotents $e_1, \ldots, e_r \in \Lambda$ such that
  $1_\Lambda = \sum_{i=1}^r e_i$. Let $J:=J(\Lambda)$. Given $i, j \in \{1, \ldots, r\}$, we pick elements $a_{ijl}
  \in e_i J e_j$ such that the residue classes $\overline a_{ijl}$ form a basis of
  $\factor{e_i J e_j}{e_i J^2 e_j}$. By general theory we have
    \[ \Ext_\Lambda^1(S_i, S_j) \isomorphic \factor{e_i J e_j}{e_i J^2 e_j}, \]
  where $S_i = \Top(\Lambda e_i)$. Let $n_{ij} := \dim_k \Ext_\Lambda^1(S_i,
  S_j)$. We define a linear map $f: k[Q_\Lambda] \to \Lambda$ as follows:
  \begin{itemize}
  \item $f(\varepsilon_i) := e_i$
  \item If $\alpha_{ijl} : i \to j$ is an arrow, then $f(\alpha_{ijl}) := a_{jil}$ for $1 \leq
    l \leq n_{ij}$.
  \item $f(p) := \prod_{j=1}^m f(\alpha_j)$ for every path $p = \alpha_1 \cdots \alpha_m$.
  \end{itemize}

  Let $\alpha: i \to j$ and $\beta: n \to m$ be arrows with $n \neq j$. Then $\beta\alpha = 0$. Then
  \[
  f(\beta)f(\alpha)
  \in e_m J e_n e_j J e_i = e_m J 0 J e_j =(0).
  \]
  This shows that $f$ is a homomorphism of algebras. We show that $\dim_k S_i = 1$. Let $P_i := \Lambda e_i$. Then $\Delta_i
  := \End_\Lambda(S_i)$ is a finite-dimensional division algebra over $k$. As $k$ is
  algebraically closed, we have $\Delta_i = k$. Now (\ref{1.4.6}) implies
    \[ 1 = \dim_k \End_\Lambda(S_i) = \dim_k\Hom_\Lambda(\Lambda, S_i) = \dim_k S_i. \]
  Wedderburn's theorem thus tells us that $\factor \Lambda {J} \isomorphic \bigoplus_{i=1}^r ke_i$.
  Moreover
  \[
  \factor{J}{J^2} \isomorphic \bigoplus_{i,j = 1}^r \factor{e_i J e_j}{e_i
    J^2 e_j}.
    \]
Consequently, the map $\overline{f} :
  \factor{k[Q_\Lambda]}{k[Q_\Lambda]_{\geq 2}} \to \factor \Lambda {J^2}$ induced by $f$ is an
  isomorphism. Hence $f(k[Q_\Lambda]) + J^2 = \Lambda$.
The remark above tells us that $f$ is surjective. By the same token $I := \ker f
  \subseteq k[Q_\Lambda]_{\geq 2}$. Since $J$ is nilpotent, there is $n$ with $J^n =
  (0)$. Hence $f(k[Q_\Lambda]_{\geq n}) \subseteq J^n = (0)$, and $k[Q_\Lambda]_{\geq n} \subseteq I$.
\end{proof}

%--------------------------------------------------------------------------------------------------------------------

\begin{definition}
  Two artin $R$-algebras  $\Lambda$ and $\Gamma$ are \textbf{Morita equivalent}\index{Morita equivalent algebras} if $\mod \Lambda$ and $\mod \Gamma$
  are equivalent $R$-categories.
\end{definition}

%--------------------------------------------------------------------------------------------------------------------

\begin{corollary}\label{3.2.8}
  Let $\Lambda$ be a finite-dimensional algebra over $k$. Then $\Lambda$ is Morita
  equivalent to $\factor{k[Q_\Lambda]}{I}$, where $k[Q_\Lambda]_{\geq n} \subseteq I \subseteq k[Q_\Lambda]_{\geq 2}$.
\end{corollary}

%--------------------------------------------------------------------------------------------------------------------

\begin{proof}
  Let $P_1, \ldots, P_n$ be a complete set of representatives for the isomorphism
  classes of the principal indecomposable $\Lambda$-modules. Then we put $P :=
  \bigoplus_{i=1}^n P_i$ and $\Gamma:=\End_\Lambda(P)^\op$. Then Corollary~\ref{3.2.5} shows that
    \[ e_P : \mod \Lambda \to \mod{\End_\Lambda(P)^\op} \]
  is an equivalence of categories.
  Then we have $\Gamma \isomorphic \bigoplus_{i=1}^ne_P(P_i)$ an isomorphism of left $\Gamma$-modules.
  Moreover $e_P(P_i)$ is projective and indecomposable with $e_P(P_i) \not\isomorphic e_P(P_j)$ for $i
  \neq j$. Hence $\Gamma$ is basic.
  In addition, the equivalence
    \[ e_P : \mod \Lambda \to \mod \Gamma \]
  is exact. It sends simples to simples and projectives to projectives. There
  result isomorphisms
    \[ \Ext_\Lambda^1(S_i, S_j) \overset{e_P}{\isomorphic} \Ext_\Gamma^1(e_P(S_i), e_P(S_j)) \]
  so that $Q_\Lambda \isomorphic Q_\Gamma$.
  Now Gabriel's theorem provides $k[Q_\Lambda]_{\geq n} \subseteq I \subseteq k[Q_\Lambda]_{\geq n}$ and an
  isomorphism $\Gamma \isomorphic \factor{k[Q_\Lambda]}{I}$.
\end{proof}

%--------------------------------------------------------------------------------------------------------------------

\section{Stable equivalence and finite representation type}

%--------------------------------------------------------------------------------------------------------------------

Throughout this section $\Lambda$ denotes an artin $R$-algebra. If $\Lambda$ is Morita equivalent to $\Gamma$, then the module categories are essentially the same. One can often transfer problems between algebras that satisfy weaker conditions.

%--------------------------------------------------------------------------------------------------------------------

\begin{definition}
Two artin $R$-algebras $\Lambda$ and $\Gamma$ are \textbf{stably equivalent}\index{stable equivalence} if there exists an equivalence $F: \Stable{\Lambda} \to \Stable{\Gamma}$.
\end{definition}

%--------------------------------------------------------------------------------------------------------------------

As before, we let $\modpf \Lambda$ be the full subcategory of $\mod \Lambda$, whose objects have no non-zero projective summands.

%--------------------------------------------------------------------------------------------------------------------

\begin{lemma}\label{3.3.1}
Let $F: \Stable \Lambda \to \Stable \Gamma$ be a functor.
\begin{enumerate}
\item The correspondence
\[
F_\pf :
\left\{
\begin{matrix}
\modpf \Lambda & \to & \modpf \Gamma \\
M & \to & F(M)_\pf
\end{matrix}
\right.
\]
satisfies $F(M) \isomorphic F_\pf(M)$ in $\Stable \Gamma$.
\item Suppose $F$ is full. Given $M,N \in \modpf \Lambda$ and $g \in \StableHom_\Gamma(F(M),
F(N))$, there exists $f \in \Hom_\Lambda(M,N)$ with
\[
F(\overline f) = g.
\]
\end{enumerate}
\end{lemma}

%--------------------------------------------------------------------------------------------------------------------

\begin{proof}
The proof of (2) is clear. So we only prove (1). Let $M \in \modpf \Lambda$. Then $F(M) \in \mod \Lambda$ and we write $F(M) = F_\pf(M) \oplus P$ with $P$ being projective. Then $F(M) \isomorphic F_\pf(M)$ in $\Stable \Gamma$.
\end{proof}

%--------------------------------------------------------------------------------------------------------------------
