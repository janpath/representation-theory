% Copyright (c) Lars Niedorf, Jan Path 2018
%
% This work is licensed under the Creative Commons Attribution-ShareAlike 4.0
% International License. To view a copy of this license, visit
% http://creativecommons.org/licenses/by-sa/4.0/ or send a letter to Creative
% Commons, PO Box 1866, Mountain View, CA 94042, USA.

% !TEX root = main.tex

% 23-11-2017

%--------------------------------------------------------------------------------------------------------------------

\begin{corollary}\label{1.6.3}
Let $I\subseteq J(\Lambda)^2$ be an ideal and $\pi:\Lambda\to\Lambda/I$ be the canonical projection. If $\Lambda=\Lambda_1\oplus\dots\oplus\Lambda_n$ is the block decomposition of $\Lambda$, then
\[
\Lambda/I=\pi(\Lambda_1)\oplus\dots\oplus\pi(\Lambda_n)
\]
is the block decomposition of $\Lambda/I$.
\end{corollary}

%--------------------------------------------------------------------------------------------------------------------

\begin{proof}
Put $\Lambda':=\Lambda/I$ and $\Lambda_i':=\pi(\Lambda_i)$. If $\Lambda_i=\Lambda e_i$ for some central primitive idempotent, then $\Lambda_i'=\Lambda' f_i$, where $f_i:=\pi(e_i)$ is a central idempotent. Moreover we have $f_if_j=\pi(e_ie_j)=0$ for $i\neq j$. Hence
\[
\Lambda' = \Lambda_1' \oplus\dots\oplus \Lambda_n'.
\]
If $f_i=f'+f''$ is a sum of orthogonal central idempotents, then Proposition~\ref{1.6.2} provides central idempotents $e',e''\in\Lambda$ such that $\pi(e')=f'$ and $\pi(e'')=f''$. Since $f'$ and $f''$ are orthogonal, we obtain $\pi(e'e'')=f'f''=0$, so that $e'e''\in I\subseteq J(\Lambda)^2$. Thus there exists $n\in\N$ such that $0=(e'e'')^n=e'e''$. Consequently $(e'+e'')^2=e'+e''$ and $\pi(e'+e'')=f'+f''=f_i=\pi(e_i)$. Now Proposition~\ref{1.6.2} tells us that $e'+e''=e_i$. This contradicts $e_i$ being primitive. Hence $f_i$ is primitive as well.
\end{proof}

%--------------------------------------------------------------------------------------------------------------------

% CHAPTER 2

\chapter{Tools from homological algebra}

%--------------------------------------------------------------------------------------------------------------------

% SECTION 2.1

\section{Schanuel's lemma}

%--------------------------------------------------------------------------------------------------------------------

Throughout, $\Lambda$ is a ring.

%--------------------------------------------------------------------------------------------------------------------

\begin{definition}
A commutative diagram
\[
\begin{tikzcd}
	M\ar[r,"f"]\ar[d,"f'"] & N\ar[d,"g"] \\
	N'\ar[r,"g'"] & L	
\end{tikzcd}
\]
of $\Lambda$-modules is called a \textbf{push-out diagram}\index{push-out diagram} if for any pair $h:N\to L'$ and $h':N'\to L'$ of $\Lambda$-linear maps such that $h\circ f = h' \circ f'$ there exists a unique $\Lambda$-linear map $\varphi$ such that $\varphi\circ g = h$ and $\varphi\circ g' = h'$. 
%
\[
\begin{tikzcd}
	M\ar[r,"f"]\ar[d,"f'"] & N\ar[d,"g"]\ar[ddrr,bend left,"h"] & \\
	N'\ar[r,"g'"]\ar[drrr,bend right,"h'"'] & L \ar[drr,dashed,"\varphi"] & \\
	   &   &   & L'
\end{tikzcd}
\]
% 
\end{definition}

%--------------------------------------------------------------------------------------------------------------------

\begin{proposition}\label{2.1.1}
Let
\[
\begin{tikzcd}
	M\ar[r,"f"]\ar[d,"f'"] & N\ar[d,"g"] \\
	N'\ar[r,"g'"] & L	
\end{tikzcd}
\]
be a commutative diagram. The following statements are equivalent:
\begin{enumerate}
\item The diagram is a push-out diagram.
\item The sequence
\[
\begin{tikzcd}
	M \ar[r,"\binom{f}{-f'}"] & N \oplus N' \ar[r,"{(g,g')}"] & L \ar[r] & (0)
\end{tikzcd}
\]
is exact.
\end{enumerate}
In either case, the induced maps
\[
\bar g : \coker f \to \coker g' \qq{and}
\bar g' : \coker f' \to \coker g
\]
are isomorphisms.
\end{proposition}

%--------------------------------------------------------------------------------------------------------------------

\begin{definition}\
\begin{enumerate}
\item We have
\[
\binom{f}{-f'}:
\left\{
\begin{matrix}
M & \to & N \oplus N' \\
m & \mapsto & (f(m),-f'(m))
\end{matrix}
\right.
\]
and
\[
(g,g'):
\left\{
\begin{matrix}
N \oplus N' & \to & L \\
(n,n') & \mapsto & g(n) + g'(n').
\end{matrix}
\right.
\]
\item The \textbf{cokernel}\index{cokernel} of a $\Lambda$-linear map $f : M \to N$ is $\coker(f):=Y/\im f$.
\end{enumerate}
\end{definition}

%--------------------------------------------------------------------------------------------------------------------

\begin{proof}\
\begin{addmargin}[1cm]{0cm}

\hspace*{-1cm}(1) $\Rightarrow$ (2): Since the diagram is commutative, we have
\[
(g,g') \circ \binom{f}{-f'}
 = g \circ f - g' \circ f'
 = 0.
\]
%
We let $L':=(N\oplus N')/\im \binom{f}{-f'}$ and consider the linear maps
\[
h :
\left\{
\begin{matrix}
N & \to & L' \\
n & \mapsto & \overline{(n,0)}
\end{matrix}
\right.
\qq{and}
h' :
\left\{
\begin{matrix}
N' & \to & L' \\
n' & \mapsto & \overline{(0,n')}.
\end{matrix}
\right.
\]
Then $(h\circ f)(m) = \overline{(f(m),0)}$ and $(h'\circ f')c(m) = \overline{(f(m),0)}$, so that $h\circ f=h'\circ f'$. Hence there is a unique $\varphi:L\to L'$ such that $\varphi\circ g =h$ and $\varphi\circ g' = h'$.
Let $(n,n')\in \ker(g,g')$. Then $g(n)+g'(n)=0$. So we have
\[
0 = \varphi(0) = \varphi\circ g(n) + \varphi\circ g'(n) = h(n) + h'(n) = \overline{(n,n')}. 
\]
Hence $(n,n')\in\im \binom{f}{-f'}$. We obtain
\[
\ker(g,g')=\im \binom{f}{-f'}.
\]
Now let $L':=\coker(g,g')=L/(g(N)+g'(N))$ and consider the canonical projection $\pi:L\to L'$. Then we have $\pi \circ g = 0 = 0 \circ g$ and $\pi \circ g' = 0 = 0 \circ g'$. By unicity, we conclude $\pi=0$. Hence $L'=(0)$ and $(g,g')$ is surjective.

For injectivity, we consider the map
\[
\bar g:
\left\{
\begin{matrix}
N/\im f & \to & L/\im g' \\
n + \im f & \mapsto & g(n) + \im g'.
\end{matrix}
\right.
\]
This is well-defined, since $g(\im f)\subseteq \im g'$. Let $\overline n \in \ker\bar g$. Then $g(n)\in \im(g')$ and there is $n'\in N'$ such that $g(n)=-g'(n')$. As result $(n,n')\in \ker(g,g')=\im\binom{f}{-f'}$. Hence there is $m\in M$ such that $n=f(m)$. Thus $\overline n = 0$. So $\bar g$ is injective.

Let $\overline x \in L/\im g'$. Since $(g,g')$ is surjective, there is $(n,n')\in N\oplus N'$ such that $g(n)+g'(n')=x$. Hence $\overline x = \overline{g(n)} = \bar g (n+\im f)$. Hence $\bar g$ is an isomorphism and the statement for $\bar g'$ follows by symmetry.

\hspace*{-1cm}(2) $\Rightarrow$ (1): Since $(g,g')$ is surjective, we have $L=g(N)+g'(N')$. Let $h:N\to L'$ and $h':N'\to L'$ be $\Lambda$-linear such that $h\circ f = h'\circ f'$. If $\varphi_1,\varphi_2:L\to L'$ with $\varphi_i\circ g=h$ and $\varphi_i\circ g'=h'$ exist, then $g(N)\subseteq \ker(\varphi_1-\varphi_2)$ and $g'(N')\subseteq \ker(\varphi_1-\varphi_2)$. Hence $L\subseteq \ker(\varphi_1-\varphi_2)$.
To prove existence, we note that $h$ and $h'$ define a map
\[
\widetilde{h} :
\left\{
\begin{matrix}
N\oplus N' & \to & L' \\
(n,n') & \mapsto & h(n) + h'(n')
\end{matrix}
\right.
\]
and observe that
\[
\widetilde{h}\circ\binom{f}{-f'}=h\circ f-h'\circ f'=0.
\]
Hence $\widetilde{h}$ vanishes on $\ker(g,g')$. Thus $\widetilde{h}$ defines a map $(N\oplus N')/\ker(g,g')\to L'$. Since $(N\oplus N')/\ker(g,g')\cong \im (g,g')=L$, there results a map $\varphi:L\to L'$ such that
\[
\varphi\circ (g,g')=\widetilde h.
\]
Consequently, $h(n) = \widetilde{h}(n,0) = \varphi\circ g(n)$ and $h'(n') = \widetilde{h}(0,n') = \varphi\circ g'(n')$.\qedhere
\end{addmargin}
\end{proof}

%--------------------------------------------------------------------------------------------------------------------

\begin{corollary}\label{2.1.2}
Given $\Lambda$-linear maps $f:M\to N'$ and $f':M'\to N'$, there is a push-out diagram
\[
\begin{tikzcd}
	M\ar[r,"f"]\ar[d,"f'"] & N\ar[d,"g"] \\
	N'\ar[r,"g'"] & L	
\end{tikzcd}
\]
\end{corollary}

%--------------------------------------------------------------------------------------------------------------------

\begin{proof}
We let $L:=N\oplus N'/\{(f(m),-f'(m))\mid m\in M\}$ and let
\[
g :
\left\{
\begin{matrix}
N & \to & L \\
n & \mapsto & \overline{(n,0)}
\end{matrix}
\right.
\qq{and}
g' :
\left\{
\begin{matrix}
N' & \to & L \\
n' & \mapsto & \overline{(0,n')}.
\end{matrix}
\right.
\]
Then $g\circ f=g'\circ f'$ and the sequence
\[
\begin{tikzcd}
	M \ar[r,"\binom{f}{-f'}"] & N \oplus N' \ar[r,"{(g,g')}"] & L \ar[r] & (0)
\end{tikzcd}
\]
is exact.\qedhere
\end{proof}














