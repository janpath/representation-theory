% Copyright (c) Lars Niedorf, Jan Path 2018
%
% This work is licensed under the Creative Commons Attribution-ShareAlike 4.0
% International License. To view a copy of this license, visit
% http://creativecommons.org/licenses/by-sa/4.0/ or send a letter to Creative
% Commons, PO Box 1866, Mountain View, CA 94042, USA.

% !TEX root = main.tex

% 07-12-2017

%--------------------------------------------------------------------------------------------------------------------

% SECTION 2.3

\section{Extensions, the Gabriel quiver and connectedness}

%--------------------------------------------------------------------------------------------------------------------

Let $\Lambda$ be an artinian ring.

\begin{definition}
Let $M \in \mod{\Lambda}$. A \textbf{projective resolution}\index{projective resolution} $(P_n,\partial_n)_{n\ge 0}$ of $M$ is an exact sequence
%
  \[ \begin{tikzcd}
      \cdots \rar["\partial_4"]&
      P_3 \rar["\partial_3"]&
      P_2 \rar["\partial_2"]&
      P_1 \rar["\partial_1"]&
      P_0 \rar["\partial_0"]& M \ar[r] & (0)
    \end{tikzcd},
    \]
%
where $P_i$ is projective. Such a resolution is said to be \textbf{minimal}\index{minimal projective resolution},
provided $(P_{n+1},\partial_{n+1})$ is a projective cover of $\ker \partial_n$
for all $n \ge 0$.
%
%\[
%  \begin{tikzcd}[column sep=small]
%    P_{n+1} \ar[rd]\ar[rr,"\partial_{n+1}"]&& P_n\\
%    &\ker \partial_n\ar[ur]&
%  \end{tikzcd}
%\]
%
\end{definition}

%--------------------------------------------------------------------------------------------------------------------

\begin{remark}
A minimal projective resolution exists if projective covers exist (as they do in $\mod{\Lambda}$). Unicity of projective covers implies that for two minimal resolutions $(P_n,\partial_n)$ and $(P_n',\partial_n')$ we obtain a commutative diagram
%
\[
  \begin{tikzcd}
    \ar[r]& P_n \dar["\wr"] \rar["\partial_n"]&
            P_{n-1} \dar["\wr"] \rar["\partial_{n-1}"]&
            \cdots \rar["\partial_1"] &
            P_0 \dar["\wr"] \rar["\partial_0"]&
            M \ar[d, equal] \ar[r] & (0) \\
    \ar[r]& P'_n \rar["{\partial'_n}"]&
            P'_{n-1} \rar["{\partial'_{n-1}}"]&
            \cdots \rar["{\partial'_1}"] &
            P'_0 \rar["{\partial'_0}"]&
            M \ar[r] & (0)
  \end{tikzcd}.
\]
%
We have $\ker \partial_n \cong \Omega^{n+1}(M)$ for all $n \ge 0$. Thus the minimal projective resolutions are isomorphic.
\end{remark}

%--------------------------------------------------------------------------------------------------------------------

\begin{definition}
Given $M, N \in \mod{\Lambda}$, the functor
$\Hom_\Lambda(-, N)$ is left exact and we obtain a complex
%
\[
  \begin{tikzcd}
    (0) \ar[r]& \Hom_\Lambda(M,N) \rar["\partial_0^*"]&
                \Hom_\Lambda(P_0,N) \rar["\partial_1^*"]&
                \Hom_\Lambda(P_1,N) \rar["\partial_2^*"]& \cdots % \ar[r]&
%                \Hom_\Lambda(P_{n-1},N) \rar["\partial_n^*"]&
%                \Hom_\Lambda(P_n,N) \ar[r]& \cdots
  \end{tikzcd}
\]
%
that is, we have $\partial_{n+1}^* \circ \partial_n^* = (\partial_n\circ
\partial_{n+1})^* = 0$. This implies that $\im \partial_n^* \subseteq \ker
\partial_{n+1}^*$.
Then
\[
\Ext^n_\Lambda(M,N) := \factor{\ker \partial_{n+1}^*}{\im \partial_n^*}
\]
is called the $n$-th \textbf{extension}\index{extension} of $M$ by $N$.
\end{definition}

%--------------------------------------------------------------------------------------------------------------------

\begin{lemma} \label{2.3.1}
Let $M \in \mod{\Lambda}$ with minimal projective resolution $(P_n)_{n \ge 0}$.
\begin{enumerate}
\item For $M \in (\mod{\Lambda})_{\mathrm{pf}}$ and simple $S\in \mod{\Lambda}$ we have $\StableHom_\Lambda(M,S) \cong \Hom_\Lambda(M,S)$. 
\item There exists a surjection \[\lambda^*: \Hom_\Lambda(\Omega^n(M),N) \to \Ext_\Lambda^n(M,N)\] with
  $
    \ker \lambda^* \subseteq \PHom_\Lambda(\Omega(M),N)$.
\item For simple $S\in \mod{\Lambda}$ we have
  \[\Ext_\Lambda^n(M,S) \cong \Hom_\Lambda(\Omega^n(M),S) \cong \Hom_\Lambda(P_n,S).\]
\end{enumerate}
\end{lemma}

%--------------------------------------------------------------------------------------------------------------------

\begin{proof}\
\begin{enumerate}
\item Let $f \in \PHom_\Lambda(M,S)$. According to Lemma~\ref{2.2.2}, there exists $g : M \to
  P_S$, such that $f = \varepsilon_S \circ g$.
\[
\begin{tikzcd}
& P_S \ar[d,"\varepsilon_S",two heads] \\
M \ar[r,"f"] \ar[ur,"g"] & S
\end{tikzcd}
\]
If $\im g \not\subseteq \Rad(P_S) =  \J{\Lambda}P_S$, then $g$ is surjective, so that $P_S$ is a direct summand of $M$. This is a contradiction since $M \in (\mod{\Lambda})_{\mathrm{pf}}$. Hence $\im g \subseteq \Rad(P_S) = \ker \varepsilon_S$, so that $f = \varepsilon_S
  \circ g = 0$.
%
\item Let $n \ge 1$. We consider the diagram
    \[
    \begin{tikzcd}[column sep=small]
      P_{n+1} \ar[rr, "\partial_{n+1}"]&& P_{n} \ar[rd, twoheadrightarrow, "\lambda"]\ar[rr]&&
      P_{n-1} \ar[rr, "\partial_{n-1}"]&& P_{n-2}\\
      &&& \Omega^n(M) \ar[ru, hook, "\gamma"]
    \end{tikzcd}.
    \]
  We thus obtain a map
    \[
    \lambda^*: \left\{ \begin{matrix}
      \Hom_\Lambda(\Omega^n(M), N) &\to &\Ext_\Lambda^n(M,N)\\
      f                   &\mapsto & [f \circ \lambda],
    \end{matrix}\right.
    \]
  where $[f \circ \lambda] = f \circ \lambda + \im \partial_n^*$ denotes the residue class.
  This is well-defined since
  \[
    \partial_{n+1}^*(f \circ \lambda) = f \circ \lambda \circ \partial_{n+1} = 0.
  \]
  Let $[g] \in \Ext_\Lambda^n(M,N)$. Then $g: P_n \to N$ satisfies $0 =
  \partial_{n+1}^*(g) = g \circ \partial_{n+1}$. Hence $g(\ker \lambda) = g(\ker
  \partial_n) = (0)$. Thus there is $f: \Omega^n(M) \to N$ such that $f \circ \lambda = g$.
  So $\lambda^*$ is surjective. For proving the second claim, we note that $f \in \Hom_\Lambda(\Omega^n(M),N)$ is in $\ker \lambda^*$ if and only if $[f \circ \lambda] = 0$. This is the case if and only if there exists $g : P_{n-1} \to N$ such that
  \[
  f \circ \lambda =
  \partial_n^*(g) = g \circ \partial_n = g \circ \gamma \circ \lambda.
  \]
If this holds, then $f = g \circ \gamma$, so that $f \in \PHom_\Lambda(\Omega^n(M), N)$.
%
\item Let $n \ge 1$. If $f \in \ker \lambda^*$, then there is $g:P_{n-1} \to S$ with $f = g \circ \gamma$. We have $\im \gamma = \ker \partial_{n-1} \subseteq \Rad(P_{n-1}) = \J{\Lambda}P_{n-1}$. Hence
  \[ f(\Omega^n) =  g \circ \lambda(\Omega^n(M)) \subseteq g( \J{\Lambda}P_{n-1}) \subseteq  \J{\Lambda}g(P_{n-1}) \subseteq  \J{\Lambda} S = (0) \]
  since $S$ is simple. Hence $\lambda^*$ is injective. Note that $\lambda$ induces an
  exact sequence
  \[
    \begin{tikzcd}
      (0) \ar[r]& \Hom_\Lambda(\Omega^n(M),S) \ar[r] & \Hom_\Lambda(P_n,S) \ar[r,
      "\partial_{n+1}^*"]& \Hom_\Lambda(P_{n+1},S)
    \end{tikzcd}.
  \]
  Let $f \in \Hom_\Lambda(P_n,S)$. Since $\im \partial_{n+1} \subseteq \J{\Lambda}P_n$, we obtain
  \[
  f \circ \partial_{n+1}(P_{n+1}) \subseteq f( \J{\Lambda}P_n) \subseteq \J{\Lambda}f(P_n) \subseteq  \J{\Lambda}S = (0).
  \]
  Hence $\partial_{n+1}^* = 0$, so that $\Hom_\Lambda(\Omega^n(M),S) \cong \Hom_\Lambda(P_n,S)$.\qedhere
\end{enumerate}
\end{proof}

%--------------------------------------------------------------------------------------------------------------------

\begin{corollary}\label{2.3.2}
  Let $\Lambda$ be a finite-dimensional, self-injective algebra over a field $k$.
  Given $M,N \in \mod{\Lambda}$, we have $\Ext_\Lambda^n(M,N) \cong \underline{\mathrm{Hom}}_\Lambda(\Omega^n(M),N)$.
\end{corollary}

%--------------------------------------------------------------------------------------------------------------------

\begin{proof}
Back to the diagram, we see that every $f \in \PHom_\Lambda(\Omega^n(M), N)$ factors through the map
$\gamma$.
\[
\begin{tikzcd}[column sep=small]
& P_{n-1} \ar[dr,"g",dashed] \\
\Omega^n(M) \ar[ur, hook, "\gamma"] \ar[rr,"f"] \ar[dr,"h"] & & N \\
& P_N \ar[ur,"\varepsilon_N",two heads] 
\end{tikzcd}.
\]
Hence $\ker \lambda^* = \PHom_\Lambda(\Omega^n(M),N)$.
\end{proof}

%--------------------------------------------------------------------------------------------------------------------

\begin{definition}
  Let $\Simples{\Lambda}$ be the set of isomorphism classes of simple $\Lambda$-modules. The
  \textbf{Gabriel quiver}\index{Gabriel quiver} $Q_\Lambda$ has $\Simples{\Lambda}$ as set of vertices. There is an arrow
  $[S] \to [T]$ if and only if $\Ext_\Lambda^1(S,T) \neq (0)$. If $\Lambda$ is a finite-dimensional
  $k$-algebra, then there are $\dim_k(\Ext_\Lambda^1(S,T))$ arrows $[S] \to [T]$. We say
  that $Q_\Lambda$ is \textbf{connected}\index{connected quiver} if the underlying graph is connected.
\end{definition}

%--------------------------------------------------------------------------------------------------------------------

\begin{example}
  Let $\Lambda = k[{\tiny \KroneckerQuiver}]$ be the path algebra of the quiver
  \[
  \begin{tikzcd}
  1 \ar[r] \ar[r,bend right] \ar[r,bend left] & 2
  \end{tikzcd}
  \]
  Then $k[{\tiny \KroneckerQuiver}]$ has underlying vector space with basis the set of all paths with
  \[
  \dim_k k[{\tiny \KroneckerQuiver}] = 5
  \]
  Products are given by concatenations.
\end{example}

%--------------------------------------------------------------------------------------------------------------------