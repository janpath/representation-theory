% Copyright (c) Lars Niedorf, Jan Path 2018
%
% This work is licensed under the Creative Commons Attribution-ShareAlike 4.0
% International License. To view a copy of this license, visit
% http://creativecommons.org/licenses/by-sa/4.0/ or send a letter to Creative
% Commons, PO Box 1866, Mountain View, CA 94042, USA.

% !TEX root = main.tex

% 27-11-2017

%--------------------------------------------------------------------------------------------------------------------

\begin{definition}
A commutative diagram
\[
\begin{tikzcd}
	L\ar[r,"g"]\ar[d,"g'"] & N\ar[d,"f"] \\
	N'\ar[r,"f'"] & M
\end{tikzcd}
\]
of $\Lambda$-modules is called a \textbf{pull-back diagram}\index{pull-back diagram} if for any pair $h:L'\to N$ and $h':L'\to N'$ of $\Lambda$-linear maps such that $f\circ h = f'\circ h'$ there exists a unique $\Lambda$-linear map $\psi$ such that $g\circ \psi = h$ and $g'\circ \psi = h'$. 
%
\[
\begin{tikzcd}
L'\ar[drrr,bend left,"h'"]\ar[ddrr,bend right,"h"']\ar[drr,dashed,"\psi"] &   &   &\\
& & L\ar[r,"g"]\ar[d,"g'"] & N\ar[d,"f"] \\
& & N'\ar[r,"f'"] & M \\
\end{tikzcd}
\]
% 
\end{definition}

%--------------------------------------------------------------------------------------------------------------------

\begin{proposition}\label{2.1.3}
Let
\[
\begin{tikzcd}
	L\ar[r,"g"]\ar[d,"g'"] & N\ar[d,"f"] \\
	N'\ar[r,"f'"] & M
\end{tikzcd}
\]
be a commutative diagram. The following statements are equivalent:
\begin{enumerate}
\item The diagram is a pull-back diagram.
\item The sequence
\[
\begin{tikzcd}
	(0) \rar  & L \ar[r,"\binom{g}{-g'}"] & N \oplus N' \ar[r,"{(f,f')}"] & M
\end{tikzcd}
\]
is exact.
\end{enumerate}
In either case, the induced maps
\[
\widetilde g : \ker g \to \ker f'
\qq{and}
\widetilde g' : \ker g' \to \ker f
\]
are isomorphisms.
\end{proposition}

%--------------------------------------------------------------------------------------------------------------------

\begin{corollary}\label{2.1.4}
Let
\[
\begin{tikzcd}
	M\ar[r,"f"]\ar[d,"f'"] & N\ar[d,"g"] \\
	N'\ar[r,"g'"] & L	
\end{tikzcd}\tag{$*$}
\]
be a commutative diagram.
\begin{enumerate}
\item If $(*)$ is a push-out diagram and $f$ is a epimorphism, then $g'$ is a epimorphism.
\item If $(*)$ is a pull-back diagram and $f$ is a monomorphism, then $g'$ is a monomorphism.
\end{enumerate}
\end{corollary}

%--------------------------------------------------------------------------------------------------------------------

\begin{lemma}\label{2.1.5}
Suppose that
\[
  \begin{tikzcd}
    (0) \rar  & M' \ar[d, "f'"] \rar["f"] &  M \ar[d, "g"] \rar["h"] & M'' \ar[d, equal] \rar  & (0) \\
    (0) \rar  & N' \rar["g'"] &  N \rar["h'"] & N'' \rar  & (0)
  \end{tikzcd}
\]
is a commutative diagram with exact rows. Then
\[
  \begin{tikzcd}
	M' \ar[d, "f'"] \rar["f"] &  M \ar[d, "g"] \\
	N' \rar["g'"] &  N
  \end{tikzcd}
\]
is a push-out diagram.
\end{lemma}

%--------------------------------------------------------------------------------------------------------------------

\begin{proof}
  We consider the sequence
  \[
    \begin{tikzcd}
      M' \ar[r,"\binom{f}{-f'}"] & M \oplus  N' \rar["{(g,g')}"] & N \rar  & (0)
    \end{tikzcd}.\tag{$*$}
  \]
  We have to show that the sequence is exact. Let $n \in N$. Then there exists $m \in M$ such that $h'(n) = h(m) = (h' \circ g) (m)$. Thus $n - g(m) \in \ker h' = \im g'$.
  Hence there is $n' \in N'$ such that $n = g(m) + g'(n')$. As a result $(g,g')$
  is surjective. Since
\[(g,g') \circ \binom{f}{-f'} = g \circ f - g' \circ f' = 0,\]
 we have $\im \binom{f}{-f'} \subseteq  \ker (g,g')$.
  Let $(m,n') \in \ker (g,g')$. Then
  \[ 0 = h' \circ (g,g')(m,n')
       = h' \circ g(m) + h' \circ g'(n')
       = h' \circ g(m). \]
  Hence  $h(m) = h' \circ g(m) = 0$, so that there is $m' \in M'$ with $m = f(m')$.
  Consequently,
  \[ 0 = g(m) + g'(n') = g\circ f(m') + g'(n')
    = g'\circ f'(m') + g'(n')
    = g'(f'(m') + n').\]
  Hence $f'(m') + n' =0$, so that $n' = -f'(m')$. As a result, $(m,n') \in \im
  \binom{f}{-f'}$ and the sequence is exact. The result follows from Proposition~\ref{2.1.1}.
\end{proof}

%--------------------------------------------------------------------------------------------------------------------

\begin{lemma}[Schanuel's lemma]\label{2.1.6}
  Let
  \[
  (0) \to N_1 \to P_1 \to M \to (0) \qq{and} (0) \to N_2 \to P_2 \to M \to (0)
  \]
  be exact sequences with $P_i$ being projective.
  Then we have an isomorphism \[N_1\oplus  P_2 \cong N_2 \oplus  P_1.\]
\end{lemma}

%--------------------------------------------------------------------------------------------------------------------

\begin{proof}
  Since $P_1$ is projective, there results a commutative diagram
  \[
    \begin{tikzcd}
      (0) \rar & N_1 \ar[d] \rar & P_1 \ar[d] \rar & M \ar[d, equal] \rar & (0)\\
      (0) \rar & N_2        \rar & P_2        \rar & M               \rar & (0)
    \end{tikzcd}
  \]
  with exact rows.
  According to Lemma~\ref{2.1.5} and Proposition~\ref{2.1.1}, this yields an exact sequence
   \[
     \begin{tikzcd}
       (0) \rar & N_1 \rar & N_2 \oplus  P_1 \rar & P_2 \rar & (0)
     \end{tikzcd}.
   \]
  As $P_2$ is projective, the sequence splits, so that $P_2\oplus N_1 \cong N_2\oplus  P_1$.
\end{proof}

%--------------------------------------------------------------------------------------------------------------------

% SECTION 2.2

\section{The Heller operator and the stable module category}

%--------------------------------------------------------------------------------------------------------------------

Throughout this section $\Lambda$ denotes an artinian ring. We let $\mod_\Lambda$ be the
category of finitely generated $\Lambda$-modules. Recall that $M$ belongs to $\mod_\Lambda$
if and only if $M$ has finite length.

Let $M \in \mod_\Lambda$. Then there exists $n \in \N$ and a surjection $\Lambda^n \twoheadrightarrow M$. 
We can therefore find a pair $(P_M,\varepsilon_M)$ consisting of a projective module $P_M$ and a surjection $\varepsilon_M: P_M \surrj M$ such that $\length{P_M}$ is minimal. In this case $(P_M, \varepsilon_M)$ is called a \textbf{projective cover}\index{projective cover} of $M$. We put $\Omega(M) := \Omega_\Lambda(M) = \ker \varepsilon_M$ and call $\Omega(M)$ a \textbf{Heller translate}\index{Heller translate} of $M$.

%--------------------------------------------------------------------------------------------------------------------

\begin{lemma}\label{2.2.1}
Let $(P_M,\varepsilon_M)$ be a projective cover of $M$.
\begin{enumerate}
\item If $(P,\varepsilon)$ consists of a projective module % (in $\mod_\Lambda$)
  and a surjection $\varepsilon: P \to M$ then there exists a surjective homomorphism
  $f: P \to P_M$ such that $\varepsilon = \varepsilon_M \circ f$.
\[
\begin{tikzcd}
& P_M \ar[d,"\varepsilon_M",two heads] \\
P \ar[ru,"f",two heads] \ar[r,"\varepsilon",two heads] & M 
\end{tikzcd}
\]
\item If $(P'_M, \varepsilon'_M)$ is another projective cover of $M$, then there exists
  an isomorphism $f:P_M \to P'_M$ such that $\varepsilon'_M \circ f = \varepsilon_M$.
\end{enumerate} 
\end{lemma}

%--------------------------------------------------------------------------------------------------------------------

\begin{proof}\
\begin{enumerate}
\item Since $P$ and $P_M$ are projective, there are linear maps $f: P
        \to P_M$ and $g: P_M \to P$ with $\varepsilon=\varepsilon_M\circ f$ and $\varepsilon_M=\varepsilon\circ g$.
        We consider $\sigma := f\circ g \in \End_\Lambda(P_M)$.
\[
\begin{tikzcd}
& P_M \ar[d,"\varepsilon_M",two heads] \ar[dl,"g"',bend right] \\
P \ar[ru,"f"'] \ar[r,"\varepsilon",two heads] & M 
\end{tikzcd}
\]        
        Then we have
        \[ \varepsilon_M \circ \sigma = \varepsilon \circ g = \varepsilon_M. \tag{$*$} \]
        Fitting's lemma yields a decomposition $P_M = P_0 \oplus P_1$
        of $P_M$ into $\sigma$-stable submodules with $\restr \sigma {P_0}$ nilpotent and
        $\sigma|_{P_1}$ bijective.
        Let $p \in P_0$. Then there is $ n\in \N$ with $\sigma^n(p) = 0$. By virtue of
        ($*$) this gives \[0 = \varepsilon_M \circ \sigma^n(p) = \varepsilon_M(p).\] Thus we get that $P_0 \subseteq  \ker \varepsilon_M$ and
        $\restr {\varepsilon_M}{P_1}$ is surjective.
        By minimality of $P_M$, we obtain $\length{P_1} = \length{P_M}$, so that
        $P_1 = P_M$. Hence $\sigma$ is an automorphism, so that $f$ is surjective.
\item By (1) there is a surjection
        $f : P_M \twoheadrightarrow P'_M$ with $\varepsilon'_M\circ f = \varepsilon_M$. Since $\length{P_M} =
        \length{P'_M}$, this map is an isomorphism.\qedhere
\end{enumerate}
\end{proof}

%--------------------------------------------------------------------------------------------------------------------

\begin{remark}
  If $(P_M,\varepsilon_M)$ and $(P'_M,\varepsilon_M)$ are projective covers of $M$, then we
  obtain a commutative diagram
    \[
      \begin{tikzcd}
        (0) \rar & \Omega_\Lambda(M) \ar[d, dashrightarrow, "\alpha"] \rar["\iota"]&  P_M \ar[d, "f"] \rar["\varepsilon_M"]& M \ar[d, equal] \rar & (0)\\
        (0) \rar & \Omega'_\Lambda(M)                            \rar["\iota'"]& P'_M            \rar["\varepsilon'_M"]& M              \rar & (0)
      \end{tikzcd}.
    \]
    We obtain $\alpha$ as follows: Let $x \in \Omega_\Lambda(M)$. Then we have $\varepsilon'_M\circ f\circ\iota(x) = \varepsilon_M \circ \iota(x) = 0$. Hence $f \circ \iota(x) \in \ker \varepsilon'_M = \im \iota'$. So we can put $\alpha(x) := (\iota')^{-1}(f \circ \iota)(x)$.
    Hence $\alpha$ is an isomorphism, so we have $\Omega_\Lambda(M) \cong \Omega'_\Lambda(M)$.
\end{remark}
