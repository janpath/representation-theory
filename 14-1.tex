% Copyright (c) Lars Niedorf, Jan Path 2018
%
% This work is licensed under the Creative Commons Attribution-ShareAlike 4.0
% International License. To view a copy of this license, visit
% http://creativecommons.org/licenses/by-sa/4.0/ or send a letter to Creative
% Commons, PO Box 1866, Mountain View, CA 94042, USA.

% !TEX root = main.tex

% 05-02-2018

%--------------------------------------------------------------------------------------------------------------------

\section{Hereditary algebras}

%--------------------------------------------------------------------------------------------------------------------


Throughout, $k$ denotes an algebraically closed field, all vector spaces over $k$ are assumed to be finite-dimensional. Let $\Lambda$ be a $k$-algebra.

%--------------------------------------------------------------------------------------------------------------------

\begin{definition}
Let $M$ be a $\Lambda$-module. A submodule $N\subseteq M$ is called \textbf{superfluous}\index{superfluous module} if for every submodule $X\subseteq M$ such that $N+X=M$ we have $X=M$. 
\end{definition}

%--------------------------------------------------------------------------------------------------------------------

\begin{lemma}\label{3.6.1}
Let $M$ be a $\Lambda$-module. The following statements are equivalent:
\begin{enumerate}
\item $N\subseteq M$ is superfluous.
\item $N\subseteq \Rad(M)$.
\end{enumerate}
\end{lemma}

%--------------------------------------------------------------------------------------------------------------------

\begin{proof}\
\begin{addmargin}[1cm]{0cm}
\hspace{-1cm}(1) $\Rightarrow$ (2): Let $X\subseteq M$ be a maximal submodule. If $N\not\subseteq X$, then $N+X=M$ so that $N$ being superfluous yields $X=M$. $\lightning$ Hence $N\subseteq X$, whence $N\subseteq \Rad(M)$.

\hspace{-1cm}(2) $\Rightarrow$ (1): Let $X\subseteq M$ be a submodule such that $M=N+X$. Suppose that $X\neq M$. Then $X$ is contained in a maximal submodule $X\subseteq Y\subseteq M$. Hence $N+X\subseteq \Rad(M)+Y\subseteq Y$. $\lightning$ \qedhere
\end{addmargin}
\end{proof}

%--------------------------------------------------------------------------------------------------------------------

\begin{theorem}\label{3.6.2}
Let $\Lambda$ be a $k$-algebra. Then the following statements hold:
\begin{enumerate}
\item If $\Lambda$ is hereditary and $I\subseteq J(\Lambda)^2$ is an ideal such that $\Lambda/I$ is hereditary, then we have $I=(0)$.
\item $\Lambda$ is hereditary if and only if $\Lambda$ is Morita equivalent to $k Q_\Lambda$.
\end{enumerate}
\end{theorem}

%--------------------------------------------------------------------------------------------------------------------

\begin{proof}\
\begin{enumerate}

\item We put $J:=J(\Lambda)$. In view of $I\subseteq J^2\subseteq J$ we have an exact sequence
\[
\begin{tikzcd}
	(0) \ar[r] & I/IJ \ar[r,"\iota"] & J/IJ \ar[r,"\pi"] & J/I \ar[r] & (0)
\end{tikzcd}
\]
of $\Lambda/I$-modules. Since $\Lambda/I$ is hereditary, $\Rad_{\Lambda/I}(\Lambda/I)=J/I$ is projective. Hence the above sequence splits. Consequently, there is a submodule $X\subseteq J/IJ$ such that
\[
J/IJ=X\oplus I/IJ
\qq{and}
X\cong J/I.
\]
On the other hand, since $I\subseteq J^2$, we have $I/IJ\subseteq\Rad_{\Lambda/I}(J/IJ)$. Thus Lemma~\ref{3.6.1} yields that $I/IJ$ is superfluous. This implies $X=J/IJ$. Hence $\ker \pi=(0)$ and thus $I=IJ$. By Nakayama's lemma, we obtain $I=(0)$.

\item Let $Q$ be a quiver without oriented cycles, so that $kQ$ is finite-dimensional. Let $S_i$ and $P_i$ be the simple $kQ$ module and the principal indecomposable module attached to the vertex $i\in Q$. Then we have an exact sequence
\[
\begin{tikzcd}
	(0) \ar[r] & \Rad(P_i) \ar[r,"\iota"] & P_i \ar[r,"\pi"] & S_i \ar[r] & (0)
\end{tikzcd}.
\]
Recall that $P_i=kQe_i$ with $e_i$ being the idempotent corresponding to $i$. Then $P_i$ has as a basis the set of all paths starting at $e_i$. Moreover $\ker(\pi)$ is generated by all paths of length $\ge 1$. Let $\{j_1,\dots,j_l\}$ be the set of immediate successors of $i$. Then $\Rad(P_i)$ may be identified with the subspace generated by all paths starting in one of the $j_t$. Hence
\[
\Rad(P_i) = \bigoplus_{t=1}^l P_{j_t}.
\]
Thus $\Rad(P_i)$ is projective and so is $\Rad kQ$. \textit{Hence} $kQ$ is hereditary.

Now let $\Lambda$ be hereditary with simple modules $S_1,\dots,S_n$ and projective covers $P_1,\dots,P_n$. In view of
$\Ext_\Lambda^1(S_i,S_j)\cong \Hom_\Lambda(\Rad(P_i),S_j)$, the preserve of an arrow $i\to j$ entails the existence of a surjective map $\varphi:\Rad(P_i)\to S_j$. Since $\Rad(P_i)$ is projective, this lifts to a surjection $\psi:\Rad(P_i)\to P_j$.
\[
\begin{tikzcd}
 & \Rad(P_i) \ar[d,two heads,"\varphi"]\ar[dl,two heads,"\psi"',dashed] \\
P_j \ar[r,two heads] & S_j
\end{tikzcd}
\]
Hence $\ell(P_j)\le \ell(\Rad(P_i))<\ell(P_i)$. Hence there are no oriented cycles in $Q_\Lambda$. Now Gabriel's theorem provides an ideal $I\subseteq (kQ_\Lambda)_{\geq 2}$ such that $\Lambda$ is Morita equivalent to $(kQ_\Lambda)/I$. Since $(kQ_\Lambda)_{\ge 1}^n=(kQ_\Lambda)_{\geq n}$ for all $n\in\N$ and since the length of all paths in $Q_\Lambda$ is bounded, we have $I\subseteq \Rad(kQ_\Lambda)=(kQ_\Lambda)_{\ge 1}$. Now apply (1).\qedhere
\end{enumerate}
\end{proof}

%--------------------------------------------------------------------------------------------------------------------

